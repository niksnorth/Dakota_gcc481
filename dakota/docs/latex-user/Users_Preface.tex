\chapter*{Preface}
\addcontentsline{toc}{chapter}{Preface}
%\chapter{Preface}

The Dakota project started in 1994 as an internal research and
development activity at Sandia National Laboratories in Albuquerque,
New Mexico. The original goal was to provide a common set of
optimization tools for a group of engineers solving structural
analysis and design problems. Prior to the Dakota project, there was
no focused effort to archive optimization methods for reuse on other
projects. Thus, engineers found themselves repeatedly building new
custom interfaces between the engineering analysis software and
optimization software. This was especially burdensome when using
parallel computing, as each project developed a unique master program
to coordinate concurrent simulations on a network of workstations or a
parallel computer. The initial Dakota toolkit provided the engineering
and analysis community at Sandia access to a variety of optimization
algorithms, hiding the complexity of the optimization software
interfaces from the users. Engineers could readily switch between
optimization software packages by simply changing a few lines in a
Dakota input file. In addition to structural analysis, Dakota has been
applied to computational fluid dynamics, nonlinear dynamics, shock
physics, heat transfer, electrical circuits, and many other science
and engineering models.

Dakota has grown significantly beyond an optimization toolkit.  In
addition to its state-of-the-art optimization methods, Dakota includes
methods for global sensitivity and variance analysis, parameter
estimation, uncertainty quantification, and verification, as well as
meta-level strategies for surrogate-based optimization, hybrid
optimization, and optimization under uncertainty. Available to all
these algorithms is parallel computation support; ranging from desktop
multiprocessor computers to massively parallel computers typically
found at national laboratories and supercomputer centers.

As of Version 5.0, Dakota is publicly released as open source under a
GNU Lesser General Public License and is available for free download
world-wide.  See \url{http://www.gnu.org/licenses/lgpl.html} for more
information on the LGPL software use agreement.  Dakota Versions 3.0
through 4.2+ were licensed under the GNU General Public License.
Dakota public release facilitates research and software collaborations
among Dakota developers at Sandia National Laboratories and other
institutions, including academic, government, and corporate
entities. See the Dakota FAQ at
\url{http://dakota.sandia.gov/faq-page} for more information on the
public release rationale and ways to contribute.

Dakota leadership includes Brian Adams (project lead), Mike Eldred
(founder and research lead), Adam Stephens (support manager), Dena
Vigil (product owner), and Jim Stewart (business manager).  For a
listing of current and former contributors and third-party library
developers, visit the Dakota webpage at
\url{http://dakota.sandia.gov}.

\textbf{Contact Information}:\\
{\small Brian M. Adams, Dakota Project Lead}\\
{\small Sandia National Laboratories}\\
{\small P.O. Box 5800, Mail Stop 1318}\\
{\small Albuquerque, NM 87185-1318}\\
{\small Web (including support information): \url{http://dakota.sandia.gov}}
