\documentclass[11pt]{article}

%\documentclass[10pt]{book}
\newcommand{\chaptermark}{1}{}

\usepackage{makeidx}
\usepackage{fancyhdr}
\usepackage{graphicx}
\usepackage{multicol}
\usepackage{float}
\usepackage{alltt}
\usepackage{doxygen}
\setcounter{tocdepth}{1}
%\makeindex
%\setlength{\footrulewidth}{0.4pt}



\setlength{\topmargin}{-.25in}
\setlength{\textheight}{9in}
\setlength{\oddsidemargin}{0.0in}
\setlength{\evensidemargin}{0.0in}
\setlength{\textwidth}{6.5in}

\setlength{\headwidth}{6.5in}
%\setlength{\footrulewidth}{0pt}
\setlength{\parindent}{3ex}
\setlength{\parsep}{0pt}

\setcounter{tocdepth}{1}

\pagestyle{plain}

\begin{document}

\lhead{}
\rhead{}
\rfoot{}
\lfoot{}
\cfoot{}
\chead{}

%%
%% Title Page
%%

\begin{center}
SANDXXXX--XXXX

Unlimited Release

Printed December XXXX
\end{center}

\vspace{0.8in}

\begin{center}
{\bf \LARGE Acro User Manual\\
[1ex]\Large Version 1.0}
\vspace*{0.4in}

William E. Hart\\
Discrete Algorithms and Mathematics Department\\
Sandia National Laboratories\\
P. O. Box 5800\\
Albuquerque, NM\\
{\tt http://www.cs.sandia.gov/$\sim$wehart}\\
{\tt wehart@sandia.gov}\\
\end{center}

\vspace*{1.0in}

\begin{quote}
{\bf\large Abstract}
This document describes the UTILIB software library.  UTILIB includes a
variety of generic components for C++ software development including
abstract data types, I/O management, sorting routines, and random number
generators.  The UTILIB library is a core component of the Acro optimization
framework, and it has been used separately for other projects at Sandia,
including DAKOTA and NETV.

\end{quote}

\vfill
\newpage

\pagestyle{fancy}
\begin{quote}
{\em acro} - From Greek akros, at the point, end, or top.\\

{\em acro} - A generic term for warblers of the genus {\em Acrocephalus}, usually refering to the sedge and/or reed warblers. The sedge warbler is a small, quite plump, warbler with a striking broad creamy stripe above its eye, and greyish brown legs. It is brown above with blackish streaks and creamy white underneath. It spends summers in the UK and winters in Africa, south of the Sahara Desert. Its song is a noisy, rambling warble compared to the more rhythmic song of the reed warbler.\\

{\em acro} - A Common Repository for Optimizers
\end{quote}

\vfill
\newpage


\pagenumbering{roman}
%\tableofcontents

\clearemptydoublepage
\pagenumbering{arabic}
\setcounter{page}{3}
\cfoot{\thepage}


\section{Introduction}


\subsection{What is PEBBL?}

PEBBL (\emph{P}arallel \emph{E}numeration and
\emph{Branch}-and-\emph{B}ound \emph{L}ibrary) is a C++ class library
for constructing serial and parallel branch-and-bound optimization
algorithms.  It is a \emph{framework}, \emph{shell}, or
\emph{skeleton} that handles the generic aspects of branch and bound,
allowing the developer to focus primarily on the unique aspects of
their particular branch-and-bound algorithm.  
It is thus similar to other software projects such
as PUBB~\cite{SHH95,SHH97}, BoB~\cite{bob95}, PPBB-Lib~\cite{PPBB96},
and ALPS~\cite{RLS04} (PEBBL's development has significantly
influenced the architecture of ALPS).  PEBBL has a number of unique
features, including very flexible parallelization strategies and the
ability to enumerate near-optimal solutions.

In principle, one can build an arbitrary branch-and-bound method atop
PEBBL by defining a relatively small number of abstract methods.  By
defining a few more methods, the algorithm can be immediately
parallelized.  PEBBL contains numerous tuning parameters that can
adapt the resulting parallel implementation to any parallel
architecture that supports an MPI message passing
library~\cite{SOHWD96}.


\subsection{The Genealogy of PEBBL}

Most of the development work on PEBBL has been carried out by
\begin{itemize}
\item Jonathan Eckstein, Rutgers University
\item William Hart, Sandia National Laboratories
\item Cynthia A. Phillips, Sandia National Laboratories.
\end{itemize}
PEBBL was originally the ``core''
layer of the PICO (\emph{P}arallel \emph{I}nteger and
\emph{C}ombina\-torial \emph{O}ptimization) package.  PICO was designed
to solve mixed integer programming problems, but included a ``core''
layer supporting implementation of arbitrary branch and bound
algorithms.  In the Spring of 2006, the development team decided to
distribute this core layer as a software package in its own right,
changing its name from ``the PICO core'' to PEBBL.  Much of PEBBL's
basic design is thus described in preliminary
publications concerning PICO~\cite{EHP97,EPH00,EPH00a}.  In fact,
significant portions of this user guide are derived
from Eckstein \emph{et al.}~\cite{EPH00a,EPH00}.

PEBBL's parallelization strategies are partially patterned after
CMMIP~\cite{Eck94,Eck97}, a parallel mixed integer programming solver
developed for the Thinking Machines CM-5 parallel supercomputer.
CMMIP, however, was specifically designed for mixed integer
programming, and to take advantage of particular features of the CM-5
architecture.  PEBBL is more generic in two senses: it is a framework that
one can use to implement any branch-and-bound algorithm, and it is
designed to run in a generic message-passing environment.  The ABACUS
package~\cite{JT98} also influenced some of PEBBL's design.


\input{packages}

\input{software}

\input{install}

\input{licenses}

\input{refs}

\begin{center}
{\Large \bf Acknowledgements}
\end{center}

A number of people have contributed to the JEGA project in various
ways.  Special thanks go out to Dr. Laura Swiler at Sandia National
Laboratories.  She is ultimately responsible for the existence of
JEGA and has provided and continues to provide valuable technical
guidance.

Thanks go out to the rest of the DAKOTA development team
who have all provided technical guidance during the development of
JEGA.


\appendix

\newpage
\input{configurationAppendix}

\newpage
\input{licensesAppendix}



%\newpage
%\bibliographystyle{abbrv}
%\bibliography{hart}

\end{document}

END-OF-DOC



